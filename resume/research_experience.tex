\cvsection{research experience}
\begin{cventries}
  \cventry
    {Research Staff Member}
    {Plasmonics Group, HGST (now Western Digital)}
    {San Jose, CA}
    {2012.06 -- 2014.12}
    {
      \begin{cvitems}
        Research scientist in charge of near-field characterization of heat assisted magnetic recording (HAMR) write heads for magnetic disk based storage devices. My research enabled scan-probe, optical, and electron microscopy and spectroscopy based characterization of the correspondence between device geometry, material composition, near-field performance, and write head efficacy and reliability.
      \end{cvitems}
    }
  \cventry
    {Postdoctoral Fellow}
    {Molecular Foundry, Lawrence Berkeley Labs}
    {Berkeley, CA}
    {2010.05 -- 2012.06}
    {
      \begin{cvitems}
        Postdoctoral researcher on a multidisciplinary project to develop first principles models for the optical and electronic structure of nanocrystals (NCs) as mediated by size and chemical environment. Experiments focused on spectroscopic characterization of the NCs by core-level x-ray and auger electron spectroscopy. I was responsible for commissioning a new multi-modal UHV surface characterization tool capable of nanometer spatial resolution Auger spectroscopy, XPS, SEM, and STM/STS.
      \end{cvitems}
    }
  \cventry
    {Graduate Student Researcher}
    {Butov Lab, UC San Diego}
    {La Jolla, CA}
    {2003.08 -- 2010.04}
    {
      \begin{cvitems}
        Conducted original research into design and implementation of experiments investigating the low temperature quantum degenerate dynamics of indirect excitons in GaAs coupled quantum well (CQW) structures. Specific studies focused on the transport characteristics of excitons within the CQW structure, coherence in the exciton system at ultra-low temperatures, and novel methods for exciton confinement and manipulation by electrostatic fields from micro/nano-meter scale contacts and patterned laser excitation. In the course of my work, I designed and implemented an optical system in the heart of our dilution refrigerator capable of one micron resolution while operating at millikelvin temperatures in magnetic fields up to 16T.
      \end{cvitems} 
    }
  \cventry
    {Research Assistant}
    {Crommie Lab, UC Berkeley}
    {Berkeley, CA}
    {2002.01 -- 2003.07}
    {
      \begin{cvitems}
        Performed experiments examining the surface structure of crystalline silicon and ad-atoms on the silicon surface using an ultrahigh vacuum variable temperature scanning tunneling microscope (STM) capable of atomic resolution. Assisted in the design of a new ultrahigh vacuum sample preparation chamber and STM system to investigate tunneling spectroscopy signatures of molecular switches.
      \end{cvitems}
    }
  \cventry
    {Research Assistant}
    {BaBar Project, Lawrence Berkeley Labs}
    {Berkeley, CA}
    {1998.09 -- 1999.10}
    {
      \begin{cvitems}
        Worked on jet finding and particle identification algorithms for the BaBar project, an experiment studying the violation of charge and parity (CP) symmetry in the decay of B mesons created in the Stanford Linear Accelerator B-Factory.
      \end{cvitems}
    }
\end{cventries}
